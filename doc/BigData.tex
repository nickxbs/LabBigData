%%% LaTeX Template
%%% This template can be used for both articles and reports.
%%%
%%% Copyright: http://www.howtotex.com/
%%% Date: February 2011

%%% Preamble
\documentclass[paper=a4, fontsize=11pt]{scrartcl}	% Article class of KOMA-script with 11pt font and a4 format


\usepackage[italian]{babel}															% English language/hyphenation
\usepackage[protrusion=true,expansion=true]{microtype}				% Better typography
\usepackage{amsmath,amsfonts,amsthm}										% Math packages
\usepackage[pdftex]{graphicx}														% Enable pdflatex
%\usepackage{color,transparent}													% If you use color and/or transparency
\usepackage[hang, small,labelfont=bf,up,textfont=it,up]{caption}	% Custom captions under/above floats
\usepackage{epstopdf}																	% Converts .eps to .pdf
\usepackage{subfig}																		% Subfigures
\usepackage{booktabs}																	% Nicer tables
\usepackage[latin1]{inputenc}
\usepackage{listings}
\usepackage{xcolor}
\lstdefinestyle{sharpc}{language=[Sharp]C, frame=lr, rulecolor=\color{blue!80!black}}

  \usepackage{courier}

%%% Advanced verbatim environment
\usepackage{verbatim}
\usepackage{fancyvrb}
\DefineShortVerb{\|}								% delimiter to display inline verbatim text


%%% Custom sectioning (sectsty package)
\usepackage{sectsty}								% Custom sectioning (see below)
\allsectionsfont{%									% Change font of al section commands
	\usefont{OT1}{bch}{b}{n}%					% bch-b-n: CharterBT-Bold font
%	\hspace{15pt}%									% Uncomment for indentation
	}

\sectionfont{%										% Change font of \section command
	\usefont{OT1}{bch}{b}{n}%					% bch-b-n: CharterBT-Bold font
	\sectionrule{0pt}{0pt}{-5pt}{0.8pt}%	% Horizontal rule below section
	}


%%% Custom headers/footers (fancyhdr package)
\usepackage{fancyhdr}
\pagestyle{fancyplain}
\fancyhead{}														% No page header
\fancyfoot[C]{\thepage}										% Pagenumbering at center of footer
\renewcommand{\headrulewidth}{0pt}				% Remove header underlines
\renewcommand{\footrulewidth}{0pt}				% Remove footer underlines
\setlength{\headheight}{13.6pt}

%%% Equation and float numbering
\numberwithin{equation}{section}															% Equationnumbering: section.eq#
\numberwithin{figure}{section}																% Figurenumbering: section.fig#
\numberwithin{table}{section}																% Tablenumbering: section.tab#



\usepackage{color}
\usepackage{xcolor}
\usepackage{listings}

 \lstset{
         basicstyle=\footnotesize\ttfamily, % Standardschrift
         %numbers=left,               % Ort der Zeilennummern
         numberstyle=\tiny,          % Stil der Zeilennummern
         %stepnumber=2,               % Abstand zwischen den Zeilennummern
         numbersep=5pt,              % Abstand der Nummern zum Text
         tabsize=2,                  % Groesse von Tabs
         extendedchars=true,         %
         breaklines=true,            % Zeilen werden Umgebrochen
         keywordstyle=\color{red},
    		frame=b,         
 %        keywordstyle=[1]\textbf,    % Stil der Keywords
 %        keywordstyle=[2]\textbf,    %
 %        keywordstyle=[3]\textbf,    %
 %        keywordstyle=[4]\textbf,   \sqrt{\sqrt{}} %
         stringstyle=\color{white}\ttfamily, % Farbe der String
         showspaces=false,           % Leerzeichen anzeigen ?
         showtabs=false,             % Tabs anzeigen ?
         xleftmargin=17pt,
         framexleftmargin=17pt,
         framexrightmargin=5pt,
         framexbottommargin=4pt,
         %backgroundcolor=\color{lightgray},
         showstringspaces=false      % Leerzeichen in Strings anzeigen ?        
 }
\lstloadlanguages{% Check Dokumentation for further languages ...
         %[Visual]Basic
         %Pascal
         %C
         %C++
         %XML
         %HTML
         %Java
{[Sharp]C}
 }
\usepackage{caption}
\DeclareCaptionFont{white}{\color{white}}
\DeclareCaptionFormat{listing}{\colorbox{gray}{\parbox{\textwidth}{#1#2#3}}}
\captionsetup[lstlisting]{format=listing,labelfont=white,textfont=white}

%%% Title	
\title{ \vspace{-1in} 	\usefont{OT1}{bch}{b}{n}
		\huge \strut Implementazione di Finding Triangles con Hadoop MapReduce\strut \\
		\Large \bfseries \strut Sistemi di elaborazione di grandi quantit\`a di dati 20013 \strut
}
\author{ 									\usefont{OT1}{bch}{m}{n}
        Nicola Febbrari\\		\usefont{OT1}{bch}{m}{n}
        Universit\`a degli Studi di Verona\\	\usefont{OT1}{bch}{m}{n}
        Facolt\`a MM.FF.NN.\\
        \texttt{nicola.febbrari@studenti.univr.it}
}
\date{13 gennaio 2014}

%%% Begin document
\begin{document}
\maketitle
\section{Introduzione}
Lo scopo del prgetto \`e quello di implementare un algoritmo per calcolare il numero di triangoli presenti in un grafo, utilizzando le tecniche di MapReduce e il Framework Haddop.


\section{Il problema}
In seguito all'enorme diffusione avuta dai social network si \`e verificata la necessit\`a di analizzare le informazioni contenute in essi. Per sviluppare un' analisi su questo tipo di informazioni si \`e utilizzata un struttura matematica che consete una semplice rappresentazione, il GRAFO.
Una caratteristica molto interessante di un grafo che rappresenta un soocial network e' il numero di triangoli contenuti in sesso. 
Il numero di questi triangoli rapportato al totale dei nodi che costituiscono il grafo è un indice di quanto sia SOCIAL il grafo che rappresenta il social network. (anzianità della comunity)
Tipicamente dati 3 nodi (A,B,C) in un grafo, dato un nodo  A che si relaziona sia con B che con C, nel grafo viene a formarsi un triangolo se esiste anche la relazione che lega B-C.

\section{Strumenti e Framework}
\paragraph{Sistema}
Apache Hadoop \`e un framework opensource utilizzato per l' elaborazione di grandi quantit\`a di dati. 
Si basa su MapReduce, un paradigma di programmazione parallela con il quale è possibile realizzare algoritmi applicabili a sitemi distributi e con alto grado di scalabilit\`a.

\paragraph{Sviluppo}
L'implementazione dell 'algoritmo di Finding Triangles  è stata realizzata scrivedo un programma Java che utilizza ed estende le Hadoop API di Base.
\paragraph{Testing}
Come suite di testing \`e stato utilizzato MRUnit una libreria Java che consente di fare Unit test sui jobs di MapReduce.
\paragraph{UI}
Come suite di supporto \`e stato utilizzato Hue, un' inerfaccia web-based per la gestione ed il monitoraggio del file sistem di Hadoop (HDFS) e dei jobs MapReduce.
\paragraph{Versioning}
Come sistema di versioning \`e stato utilizzato GitHub, un servizio web che utilizza la piattaforma GIT.
\paragraph{Sorgente dati}
Come sistema di versioning \`e stato utilizzato GitHub, un servizio web che utilizza la piattaforma GIT.




\section{Implementazioni}
\paragraph{Algoritmo 2 Jobs}
\paragraph{Algoritmo 1 Jobs}




\paragraph{Ottimizzazioni}
Ad ogni turno l'algoritmo decide quale sia la mossa migliore. Partendo da una situazione consolidata della Tavola, vengono elaborati i valori di ogni mossa possibile usando un calcolo ricorsivo sugli stati successivi generati da tutte le possibili combinazioni di mosse valide.\\
La ricorsione percorre l'albero fino alle foglie, valuta il valore della foglia assegnandogli un +1 se la foglia rappresenta uno stato di vittoria, 0 se di pareggio o -1 in caso di sconfitta. 
Dopo aver calcolato il valore di una foglia viene fatto \textit{ backtracking} sull'albero di gioco eventualmente aggiornando la scelta del nodo ottimo \textit{ MinMax} in base al livello dell'albero (nel caso di turno proprio turno Max nel caso di turno dell'avversario Min).

\section{Complessit\`a}
\paragraph{Algoritmo 2 Jobs}
\paragraph{Algoritmo 1 Jobs}


\section{Osservazioni}

\end{document}